\subsection*{Objetivo}

O objetivo deste trabalho é implementar um programa para calcular a solução discreta para uma Equação Diferencial Parcial com duas variáveis indepententes utilizando Diferenças Finitas centrais de primeira ordem e o método de Gauss-\/\+Seidel. \subsection*{Especificação}

O trabalho consiste em calcular a solução discreta, utilizando Diferenças Finitas Centrais e o Método de Gauss-\/\+Seidel, para a seguinte Equação Diferencial Parcial\+:  onde\+:   O domínio é definido por\+:  Nas fronteiras\+:    

A discretização do domínio deve ser feita em uma malha com espaçamento entre os pontos (0$<$hx,hy$<$pi) calculados a partir do número de pontos em cada dimensão () dados como parâmetro do programa. \begin{DoxyVerb}Escolha uma estrutura de dados eficiente para representar o Sistema Linear resultante;
Escolha um layout eficiente para as variáveis e termos independentes do seu sistema;
Use um vetor nulo como estimativa inicial para a solução;
\end{DoxyVerb}


\subsection*{Execução}

O pacote de software a ser construído deve gerar um executável chamado pde\+Solver, que deve ser invocado da seguinte forma\+: ./pde\+Solver -\/nx $<$nx$>$ -\/ny $<$ny$>$ -\/i $<$max\+Iter$>$ -\/o arquivo\+\_\+saida Onde\+:

-\/nx\+: parâmetro obrigatório definindo o número de pontos a serem calculados na dimensão X. -\/ny\+: parâmetro obrigatório definindo o número de pontos a serem calculados na dimensão Y. -\/i max\+Iter\+: parâmetro obrigatório definindo o número de iterações a serem executadas. -\/o arquivo\+\_\+saida\+: parâmetro no qual arquivo\+\_\+saida é o caminho completo para o arquivo que vai conter a solução (valores da função em cada ponto da grade). Caso este parâmetro não esteja especificado, a saída deve ser stdout Esta solução deve estar formatada para servir de entrada ao comando gnuplot, de forma que ele possa automaticamente gerar o gráfico da função. Além disso, no início do arquivo, deve constar sob a forma de comentários do gnuplot\+: O tempo médio de execução de cada iteração do Método de Gauss-\/\+Seidel O valor do resíduo para cada iteração.

\subparagraph*{}

\section*{Tempo Método GS\+: $<$média de tempo para o cálculo de uma iteração do método, em milisegundos$>$}

\# \section*{Norma L2 do Residuo}

\section*{i=1\+: }

\section*{i=2\+: }

\section*{i=3\+: }

\section*{...}

\subparagraph*{}

\begin{DoxyVerb}Tempo Método GS: deve ser calculado em milisegundos, utilizando-se a função timestamp() especificada aqui. O tempo é calculado a partir do início da iteração do método até a obtenção do vetor solução daquela iteração. O resultado deve ser a média aritmética do tempo de todas iterações.
\end{DoxyVerb}


\subsection*{Makefile}

O arquivo Makefile deve possuir as regras necessárias para compilar os módulos individualmente e gerar o programa executável. As seguintes regras devem existir O\+B\+R\+I\+G\+A\+T\+O\+R\+I\+A\+M\+E\+N\+TE\+: \begin{DoxyVerb}all: compila e produz um executável chamado pdeSolver no diretório login1-login2/;
clean: remove todos os arquivos temporários e os arquivos gerados pelo Makefile (*.o, executável, etc.).
doc: gera a documentação Doxygen em formato html\end{DoxyVerb}
 